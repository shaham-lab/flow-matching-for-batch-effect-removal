\section{Stable Flow Matching for Batch Effect Removal (SFMBER)}

The SFMBER framework addresses the challenge of batch correction by  \emph{Conditional Flow Matching (CFM)} within a learned latent manifold, the framework ensures that the transport of distributions is both computationally efficient and biologically stable. take insperation from stable difussion architecture , the process begins by training an VAE to map high-dimensional measurements $x \in \mathcal{X}$ into a shared latent space $\mathcal{Z}$ we use scGen as a vae.

Rather than performing distribution matching in the high-dimensional input space, which is often sparse and noisy, SFMBER learns a time-dependent vector field $v_\theta$ directly on the latent manifold that created by scGen. For a source batch $B_0$ and a target batch $B_1$, we derive the latent sets $Z_0 = \{E(x_{0,i})\}_{i=1}^m$ and $Z_1 = \{E(x_{1,j})\}_{j=1}^m$. Following a \emph{Minibatch Optimal Transport} approach, we solve for an empirical coupling $\boldsymbol{\pi}^*$ to identify paired latent points $(z_0, z_1)$ that define the most parsimonious transport paths. We define a conditional probability path as the linear interpolation $\psi_t(z_0, z_1) = (1-t)z_0 + t z_1$, which corresponds to the optimal transport displacement interpolation. 



The vector field $v_\theta$ is then optimized by minimizing the latent CFM objective:
\begin{equation}
\mathcal{L}_{\mathrm{CFM}}(\theta) = \mathbb{E}_{t \sim \mathcal{U}(0,1), (z_0, z_1) \sim \boldsymbol{\pi}^*} \left\| v_\theta(\psi_t(z_0, z_1), t) - (z_1 - z_0) \right\|^2.
\end{equation}
By training on OT-coupled pairs, the model learns to transport $Z_0$ to $Z_1$ via straight-line trajectories. This significantly stabilizes training and prevents the "path crossing" phenomenon that often degrades the performance of standard generative models in single-cell analysis.

At inference time, a source sample $x_0 \in B_0$ is corrected by first mapping it to the latent space $z_0 = E(x_0)$. The batch-corrected latent representation $z_{\text{corr}}$ is then obtained by numerically integrating the learned vector field:
\begin{equation}
z_{\text{corr}} = z_0 + \int_0^1 v_\theta(z_\tau, \tau) d\tau, \quad \text{where } \frac{dz_t}{dt} = v_\theta(z_t, t).
\end{equation}
Finally, the corrected measurement in the original input space is generated via $x_{\text{corr}} = D(z_{\text{corr}})$. This three-step approach ensures that while the global distribution is aligned to match the target batch $B_1$, the local biological identity and manifold structure of the individual cell are preserved through the continuity of the flow in $\mathcal{Z}$.
